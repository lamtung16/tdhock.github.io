\documentclass[margin,line]{res}
\usepackage{verbatim}
\usepackage{url}
\urlstyle{same}

\oddsidemargin -.5in
\evensidemargin -.5in
\textwidth=6.0in
\itemsep=0in
\parsep=0in

\newenvironment{list1}{
  \begin{list}{\ding{113}}{%
      \setlength{\itemsep}{0in}
      \setlength{\parsep}{0in} \setlength{\parskip}{0in}
      \setlength{\topsep}{0in} \setlength{\partopsep}{0in} 
      \setlength{\leftmargin}{0.17in}}}{\end{list}}
\newenvironment{list2}{
  \begin{list}{$\bullet$}{%
      \setlength{\itemsep}{0in}
      \setlength{\parsep}{0in} \setlength{\parskip}{0in}
      \setlength{\topsep}{0in} \setlength{\partopsep}{0in} 
      \setlength{\leftmargin}{0.2in}}}{\end{list}}

\begin{document}

\name{
\begin{tabular*}{7.4in} {@{\extracolsep{\fill}}lr}
Toby Dylan Hocking & Curriculum Vitae 
\end{tabular*}
}

\begin{resume}
\section{\sc Contact Information}
\vspace{.05in}
\begin{tabular*}{6.1in} {@{\extracolsep{\fill}}ll}
 Mail and Courier: & Citizenship: USA \\            
 7260 Boyer \#2  & Phone:  (514) 398-3311 x 00721 \\   
Montreal, QC & E-mail:  toby.hocking@mail.mcgill.ca       \\
H2R 2R7 Canada &  Webpage: \url{http://tdhock.github.io}\\     
\end{tabular*}


\section{\sc Research Interests}

Supervised and unsupervised machine learning models for large
datasets, based on convex and discrete optimization (regression,
classification, ranking, clustering, changepoint detection, survival
analysis). Applications to genomics, neuroscience,
audio, internet, sensors, recommendation systems.

% Statistical methods for large datasets, regression, classification,
% ranking, changepoint detection, survival analysis; applications in genomics, 

\section{\sc Education and Professional Experience}

{\bf McGill University}, Montreal, Canada (2014-2017).\\
\vspace*{-.1in}
\begin{list2}
\item[] Postdoc with Guillaume Bourque, Department of Human Genetics.
\item[]``Changepoint detection and regression models for peak detection in genomic data.''
\end{list2}

{\bf Tokyo Institute of Technology}, Tokyo, Japan (2013).\\
\vspace*{-.1in}
\begin{list2}
\item[] Postdoc with Masashi Sugiyama, Department of Computer Science.
\item[] ``Support vector machines for ranking and comparing.''
\end{list2}

{\bf \'{E}cole Normale Sup\'{e}rieure}, Cachan, France (2009-2012).\\
\vspace*{-.1in}
\begin{list2}
\item[] Ph.D. with Francis Bach, D\'{e}partement d'Informatique; Jean-Philippe Vert, Institut Curie.
\item[] ``Learning algorithms and statistical software, with applications to bioinformatics."
\end{list2}

{\bf Universit\'e Paris 6}, Paris, France (2008-2009).\\
\vspace*{-.1in}
\begin{list2}
\item[] Master of Statistics, internship at INRA with Mathieu Gautier and Jean-Louis Foulley.
\item[] ``A Bayesian Outlier Criterion to Detect SNPs under Selection in Large Data Sets."
\end{list2}

{\bf Sangamo BioSciences}, Richmond, CA, USA (2006-2008).\\
\vspace*{-.1in}
\begin{list2}
\item[] Research Assistant with Jeff Miller in the Technology group.
\item[] ``A web app for visualization and statistical analysis of experimental data.''
\end{list2}

{\bf University of California, Berkeley}, CA, USA (2002-2006).\\
\vspace*{-.1in}
\begin{list2}
\item[] Double B.A. in Statistics, Molecular and Cell Biology; thesis in Statistics with Terry Speed.
\item[] ``Chromosomal copy number analysis using SNP microarrays and a binomial test statistic.'' 
\end{list2}

\section{\sc Refereed Publications}

In addition to peer-reviewed journals, I publish papers at
machine learning conferences such as ICML and NIPS, which have
double-blind peer reviews, and only accept about 20\% of submitted
papers.

{\bf Hocking TD}, Goerner-Potvin P, Morin A, Shao X, Pastinen T,
Bourque G. Optimizing ChIP-seq peak detectors using visual labels and
supervised machine learning. {\it Bioinformatics} (2017) 33 (4): 491-499.

Shimada K, Shimada S, Sugimoto K, Nakatochi M, Suguro M, Hirakawa A,
{\bf Hocking TD}, Takeuchi I, Tokunaga T, Takagi Y, Sakamoto A, Aoki T, Naoe
T, Nakamura S, Hayakawa F, Seto M, Tomita A, Kiyoi H. Development and
analysis of patient-derived xenograft mouse models in intravascular
large B-cell lymphoma. {\it Leukemia} 2016.

Chicard M, Boyault S, Colmet-Daage L, Richer W, Gentien D, Pierron G,
Lapouble E, Bellini A, Clement N, Iacono I, Bréjon S, Carrere M, Reyes
C, {\bf Hocking TD}, Bernard V, Peuchmaur M, Corradini N, Faure-Conter
C, Coze C, Plantaz D, Defachelles A-S, Thebaud E, Gambart M, Millot F,
Valteau-Couanet D, Michon J, Puisieux A, Delattre O, Combaret V,
Schleiermacher G. Genomic copy number profiling using circulating free
tumor DNA highlights heterogeneity in neuroblastoma. {\it Clinical Cancer
Research} 2016.

Maidstone R, {\bf Hocking TD}, Rigaill G, Fearnhead P. On optimal
multiple changepoint algorithms for large data. {\it Statistics and
Computing} (2016). doi:10.1007/s11222-016-9636-3 

{\bf Hocking TD}, Rigaill G, Bourque G. PeakSeg: constrained optimal
segmentation and supervised penalty learning for peak detection in
count data. {\it International Conference on Machine Learning (ICML)},
2015.

Suguro M, Yoshida N, Umino A, Kato H, Tagawa H, Nakagawa M, Fukuhara
N, Karnan S, Takeuchi I, {\bf Hocking TD}, Arita K, Karube K, Tsuzuki
S, Nakamura S, Kinoshita T, Seto M. Clonal heterogeneity of lymphoid
malignancies correlates with poor prognosis. {\it Cancer Sci.} 2014
Jul;105(7):897-904.

{\bf Hocking TD}, Boeva V, Rigaill G, Schleiermacher G,
Janoueix-Lerosey I, Delattre O, Richer W, Bourdeaut F, Suguro M, Seto
M, Bach F, Vert J-P. SegAnnDB: interactive Web-based genomic
segmentation. {\it Bioinformatics} (2014) 30 (11):
1539-1546. DOI:10.1093/bioinformatics/btu072

{\bf Hocking TD}, Wutzler T, Ponting K and Grosjean P. Sustainable,
extensible documentation generation using inlinedocs. {\it Journal of
Statistical Software} (2013), 54(6), 1-20. DOI:10.18637/jss.v054.i06

{\bf Hocking TD}, Schleiermacher G, Janoueix-Lerosey I, Boeva V, Cappo
J, Delattre O, Bach F, Vert J-P. Learning smoothing models of copy
number profiles using breakpoint annotations. {\it BMC Bioinfo.} 2013,
14:164. DOI:10.1186/1471-2105-14-164

{\bf Hocking TD}, Rigaill G, Bach F, Vert J-P. Learning sparse
penalties for change-point detection using max-margin interval
regression. {\it International Conference on Machine Learning (ICML)}, 2013.

{\bf Hocking TD}, Joulin A, Bach F, Vert J-P. Clusterpath: an
Algorithm for Clustering using Convex Fusion Penalties. {\it International Conference on Machine Learning (ICML)}, 2011.

Gautier M, {\bf Hocking TD}, Foulley JL. A Bayesian outlier criterion
to detect SNPs under selection in large data sets. {\it PloS ONE} 5
(8), e11913 (2010).

Doyon Y, McCammon JM, Miller JC, Faraji F, Ngo C, Katibah GE, Amora R,
{\bf Hocking TD}, Zhang L, Rebar EJ, Gregory PD, Urnov FD, Amacher
SL. Heritable targeted gene disruption in zebrafish using designed
zinc-finger nucleases. {\it Nature biotechnology} 26 (6), 702-70
(2008).

\section{\sc Papers in Progress}

{\bf Hocking TD}, Rigaill G, Fearnhead P, Bourque G. A log-linear
segmentation algorithm for peak detection in genomic data. Preprint
arXiv:1703.03352. Under review at {\it Annals of Applied Statistics}.

Drouin A, {\bf Hocking TD}, Laviolette F. Max margin interval
trees. Under review at {\it Neural Information Processing Systems (NIPS)}.

Venuto D, Spanurattana S, Sugiyama M, {\bf Hocking TD}. Support vector
comparison machines. Preprint arXiv:1401.8008. 

Sievert C, Cai J, VanderPlas S, Khan F, {\bf Hocking TD}. Extending ggplot2's
grammar of graphics implementation for linked and dynamic graphics on
the web.

{\bf Hocking TD}, Khare A. Learning penalty functions for changepoint
detection using elastic-net regularized accelerated failure time
models.

Narahara M, {\bf Hocking TD}, Bourque G, Yamada R, Setoh K, Matsuda F,
Lathrop M. Transcriptomic analysis of antibody responses to seasonal
influenza vaccine reveals predictive gene signatures and potential key
transcription factors.

Alirezaie N, Majewski J, {\bf Hocking TD}. A supervised machine
learning method for predicting pathogenicity of genetic variants.

\section{\sc Lightly- and Non-Refereed Publications}

{\bf Hocking TD}. A breakpoint detection error function for
segmentation model selection and validation. Preprint
arXiv:1509.00368.

{\bf Hocking TD}, Bourque G. PeakSegJoint: fast supervised peak
detection via joint segmentation of multiple count data
samples. Preprint arXiv:1506.01286.

{\bf Hocking TD}, Rigaill G. SegAnnot: an R package for fast
segmentation of annotated piecewise constant signals, Preprint
hal-00759129.

\section{\sc Conference Tutorials}

{\bf Hocking TD}, Killick R. Introduction to optimal changepoint detection algorithms, {\it useR} 2017.

{\bf Hocking TD}, Ekstr\o m CT. Understanding and creating interactive
graphics, {\it useR} 2016.

\section{\sc Invited talks (selected)}

McGill Barbados epigenomics workshop, 2015.

Joint statistical meeting (JSM) 2015 (declined; my collaborator Susan
VanderPlas gave the talk instead).

Workshop on Machine Learning and Applications to Biology, MLAB Sapporo
2013.

Google Research, New York, 2012.

Universit\'e Rennes, 2012.

Universit\'e Angers, 2012.

INRIA (French computer science research institute), Lille, 2012.

Institut de Biologie de Lille 2011.

\section{\sc Honors and Awards (Selected)}

``Mobilit\'e entrant'' travel award to work with Guillem Rigaill in
Universit\'e Evry, France, 2016.

International useR conference, Best Student Poster Award, ``Adding
direct labels to plots,'' 2011.

INRIA/INRA (French computer science and agricultural research institutes), Ph.D. scholarship, 2009 (declined).

UC Berkeley, department of Statistics VIGRE research scholarship, 2001.

UC Berkeley, Cal Band George Miller scholarship, 2000.

\section{\sc Professional Service}

President of organizing committee for ``R in Montreal 2018'' conference.

Co-administrator and mentor for R project in Google Summer of Code.

Reviewer: International Conference on Machine Learning (ICML),
Advances in Neural Infromation Processing Systems (NIPS), Journal of
Machine Learning Research (JMLR), Artificial Intelligence Review,
Journal of Computational and Graphical Statistics (JCGS), R Journal,
Bioinformatics, PLOS Computational Biology, BMC Bioinformatics, IEEE
Transactions on Pattern Analysis and Machine Intelligence.

\section{\sc Software Online (Selected)} 

{\bf R}: clusterpath, directlabels, animint, plotly, sublogo,
inlinedocs, quadmod, bams, neuroblastoma, breakpointError, SegAnnot,
rankSVMcompare, gganim, animint, requireGitHub, WeightedROC, revector,
PeakError, PeakSegDP, PeakSegJoint, PeakSegOptimal, memtime,
namedCapture, fpop, \mbox{penaltyLearning}, mmit, iregnet.

{\bf Python}: str.extractall in pandas, annotate\_regions, SegAnnDB.

\begin{comment}
\pagebreak

\section{\sc References}

\noindent {\bf Francis Bach}\\
Scientific Leader,
SIERRA Project-Team\\
%Department of Computer Science\\
Institut National de Recherche en Informatique et en Automatique\\
%Phone: (604)822-9878\\
E-mail: francis.bach@ens.fr\\
Webpage: \url{http://www.di.ens.fr/~fbach}

\noindent {\bf Michael Friedlander}\\
Associate Professor,
Department of Computer Science\\
University of British Columbia\\
%Phone: (604)822-9878\\
E-mail: mpf@cs.ubc.ca\\
Webpage: http://www.cs.ubc.ca/$\sim$mpf

\noindent {\bf Kevin Murphy}\\
Research Scientist,
%Department of Computer Science, Department of Statistics\\
%University of British Columbia\\
Google\\
%Phone: (604)822-9878\\
E-mail: murphyk@gmail.com\\
Webpage: http://www.cs.ubc.ca/$\sim$murphyk


%\noindent {\bf R\'{o}mer Rosales}\\
%Research Scientist\\
%CAD and Knowledge Systems\\
%Siemens Medial Solutions USA, Inc.\\
%%Phone: (610)448 4793\\
%E-mail: romer.rosales@gmail.com\\
%Webpage: http://people.csail.mit.edu/romer/\\
%
%\noindent {\bf Glenn Fung}\\
%Research Scientist\\
%CAD and Knowledge Systems\\
%Siemens Medial Solutions USA, Inc.\\
%E-mail: glenn.fung@siemens.com\\
%Webpage: http://pages.cs.wisc.edu/$\sim$gfung/\\
%
\noindent {\bf Russell Greiner}\\
 Professor, Department of Computing Science\\
University of Alberta\\
%Phone: (604)822-9878\\
E-mail: greiner@cs.ualberta.ca\\
Webpage: http://webdocs.cs.ualberta.ca/$\sim$greiner/
\end{comment}
\end{resume}
\end{document}




