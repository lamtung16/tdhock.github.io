\documentclass[margin,line]{res}
\usepackage{verbatim}
\usepackage{url}
\urlstyle{same}

\usepackage[utf8]{inputenc}
\usepackage[T1]{fontenc}

\oddsidemargin -.5in
\evensidemargin -.5in
\textwidth=6.0in
\itemsep=0in
\parsep=0in

\newenvironment{list1}{
  \begin{list}{\ding{113}}{%
      \setlength{\itemsep}{0in}
      \setlength{\parsep}{0in} \setlength{\parskip}{0in}
      \setlength{\topsep}{0in} \setlength{\partopsep}{0in} 
      \setlength{\leftmargin}{0.17in}}}{\end{list}}
\newenvironment{list2}{
  \begin{list}{$\bullet$}{%
      \setlength{\itemsep}{0in}
      \setlength{\parsep}{0in} \setlength{\parskip}{0in}
      \setlength{\topsep}{0in} \setlength{\partopsep}{0in} 
      \setlength{\leftmargin}{0.2in}}}{\end{list}}

\begin{document}

\name{
\begin{tabular*}{7.4in} {@{\extracolsep{\fill}}lr}
Toby Dylan Hocking & Curriculum Vitae 
\end{tabular*}
}

\begin{resume}
\section{\sc Contact}
\vspace{.05in}
\begin{tabular*}{6.1in} {@{\extracolsep{\fill}}ll}
 Courrier : & Citoyenneté : Etats-Unis \\            
 7260 Boyer \#2  & Tél :  (514) 398-3311 x 00721 \\   
Montreal, QC & Courriel :  toby.hocking@mail.mcgill.ca       \\
H2R 2R7 Canada &  Page web : \url{http://tdhock.github.io}\\     
\end{tabular*}


\section{\sc Expertise de Recherche}

% Statistical machine learning models for large
% datasets, based on convex and discrete optimization (regression,
% classification, ranking, clustering, changepoint detection, survival
% analysis). Applications to genomics, neuroscience,
% audio, internet, sensors, recommendation systems.
Modèles statistiques et algorithmes d'apprentissage pour des grands
jeux de données. Optimisation convexe et discrète (régression,
classification, classement, regroupement, détection de
ruptures). Applications à la génomique, la neuroscience, l'audio,
internet, sondes, systèmes de recommendation.

% Statistical methods for large datasets, regression, classification,
% ranking, changepoint detection, survival analysis; applications in genomics, 

\section{\sc Experiences professionnelles}

{\bf McGill University}, Montreal, Canada (2014-).\\
\vspace*{-.1in}
\begin{list2}
\item[] Postdoc avec Guillaume Bourque, Département de génétique humaine.
\item[]« Apprentissage automatique pour les données ChIP-seq »
\end{list2}

{\bf Tokyo Institute of Technology}, Tokyo, Japan (2013).\\
\vspace*{-.1in}
\begin{list2}
\item[] Postdoc avec Masashi Sugiyama, Department d'Informatique.
\item[] « Séparateur à vaste marge pour classement et comparaison »
\end{list2}

{\bf Sangamo BioSciences}, Richmond, CA, USA (2006-2008).\\
\vspace*{-.1in}
\begin{list2}
\item[] Research Assistant avec Jeff Miller.
\item[] « Un application web pour l'analyse de données »
\end{list2}

\section{\sc Formation}

{\bf \'{E}cole Normale Sup\'{e}rieure}, Cachan, France (2009-2012).\\
\vspace*{-.1in}
\begin{list2}
\item[] Doctorat avec Francis Bach, D\'{e}partement d'Informatique; Jean-Philippe Vert, Institut Curie.
\item[] « Algorithmes d'apprentissage avec applications à la bioinformatique »
\end{list2}

{\bf Universit\'e Paris 6}, Paris, France (2008-2009).\\
\vspace*{-.1in}
\begin{list2}
\item[] Master de Statistiques, stage chez INRA avec Mathieu Gautier and Jean-Louis Foulley.
\item[] « Un modèle bayesien pour détecter la séléction génétique »
\end{list2}

{\bf University of California, Berkeley}, CA, USA (2002-2006).\\
\vspace*{-.1in}
\begin{list2}
\item[] Double Bachelor en Biologie et Statistiques; thèse en Statistiques avec Terry Speed.
\item[] « Analyse de nombre de copie de chromosomes avec la distribution binomiale » 
\end{list2}

\section{\sc Prix}

Prix « mobilit\'e entrant » pour travailler avec Guillem Rigaill à
l'Universit\'e Evry, France, 2016.

Congrès useR, Best Student Poster, « Adding
direct labels to plots » 2011.

INRIA/INRA, concours de doctorat, 2009 (refusé).

UC Berkeley, departement de Statistiques, prix VIGRE pour la recherche, 2001.

UC Berkeley, prix George Miller, 2000.

\section{\sc Publications \\avec comité\\ de lecture}

En plus de publier dans les journaux, je publie des papiers dans les
congrès d'apprentissage automatique tels que ICML et NIPS, qui ont des comités de lecture doublement aveugle. Ils acceptent seulement 20\% des papiers soumis.

Drouin A, {\bf Hocking TD}, Laviolette F. Max margin interval
trees. {\it Neural Information Processing Systems (NIPS)}, 2017.

{\bf Hocking TD}, Goerner-Potvin P, Morin A, Shao X, Pastinen T,
Bourque G. Optimizing ChIP-seq peak detectors using visual labels and
supervised machine learning. {\it Bioinformatics} (2017) 33 (4): 491-499.

Shimada K, Shimada S, Sugimoto K, Nakatochi M, Suguro M, Hirakawa A,
{\bf Hocking TD}, Takeuchi I, Tokunaga T, Takagi Y, Sakamoto A, Aoki T, Naoe
T, Nakamura S, Hayakawa F, Seto M, Tomita A, Kiyoi H. Development and
analysis of patient-derived xenograft mouse models in intravascular
large B-cell lymphoma. {\it Leukemia} 2016.

Chicard M, Boyault S, Colmet-Daage L, Richer W, Gentien D, Pierron G,
Lapouble E, Bellini A, Clement N, Iacono I, Bréjon S, Carrere M, Reyes
C, {\bf Hocking TD}, Bernard V, Peuchmaur M, Corradini N, Faure-Conter
C, Coze C, Plantaz D, Defachelles A-S, Thebaud E, Gambart M, Millot F,
Valteau-Couanet D, Michon J, Puisieux A, Delattre O, Combaret V,
Schleiermacher G. Genomic copy number profiling using circulating free
tumor DNA highlights heterogeneity in neuroblastoma. {\it Clinical Cancer
Research} 2016.

Maidstone R, {\bf Hocking TD}, Rigaill G, Fearnhead P. On optimal
multiple changepoint algorithms for large data. {\it Statistics and
Computing} (2016). doi:10.1007/s11222-016-9636-3 

{\bf Hocking TD}, Rigaill G, Bourque G. PeakSeg: constrained optimal
segmentation and supervised penalty learning for peak detection in
count data. {\it International Conference on Machine Learning (ICML)},
2015.

Suguro M, Yoshida N, Umino A, Kato H, Tagawa H, Nakagawa M, Fukuhara
N, Karnan S, Takeuchi I, {\bf Hocking TD}, Arita K, Karube K, Tsuzuki
S, Nakamura S, Kinoshita T, Seto M. Clonal heterogeneity of lymphoid
malignancies correlates with poor prognosis. {\it Cancer Sci.} 2014
Jul;105(7):897-904.

{\bf Hocking TD}, Boeva V, Rigaill G, Schleiermacher G,
Janoueix-Lerosey I, Delattre O, Richer W, Bourdeaut F, Suguro M, Seto
M, Bach F, Vert J-P. SegAnnDB: interactive Web-based genomic
segmentation. {\it Bioinformatics} (2014) 30 (11):
1539-1546. DOI:10.1093/bioinformatics/btu072

{\bf Hocking TD}, Wutzler T, Ponting K and Grosjean P. Sustainable,
extensible documentation generation using inlinedocs. {\it Journal of
Statistical Software} (2013), 54(6), 1-20. DOI:10.18637/jss.v054.i06

{\bf Hocking TD}, Schleiermacher G, Janoueix-Lerosey I, Boeva V, Cappo
J, Delattre O, Bach F, Vert J-P. Learning smoothing models of copy
number profiles using breakpoint annotations. {\it BMC Bioinfo.} 2013,
14:164. DOI:10.1186/1471-2105-14-164

{\bf Hocking TD}, Rigaill G, Bach F, Vert J-P. Learning sparse
penalties for change-point detection using max-margin interval
regression. {\it International Conference on Machine Learning (ICML)}, 2013.

{\bf Hocking TD}, Joulin A, Bach F, Vert J-P. Clusterpath: an
Algorithm for Clustering using Convex Fusion Penalties. {\it International Conference on Machine Learning (ICML)}, 2011.

Gautier M, {\bf Hocking TD}, Foulley JL. A Bayesian outlier criterion
to detect SNPs under selection in large data sets. {\it PloS ONE} 5
(8), e11913 (2010).

Doyon Y, McCammon JM, Miller JC, Faraji F, Ngo C, Katibah GE, Amora R,
{\bf Hocking TD}, Zhang L, Rebar EJ, Gregory PD, Urnov FD, Amacher
SL. Heritable targeted gene disruption in zebrafish using designed
zinc-finger nucleases. {\it Nature biotechnology} 26 (6), 702-70
(2008).

\section{\sc Projets \\en cours}

{\bf Hocking TD}, Rigaill G, Fearnhead P, Bourque G. A log-linear
segmentation algorithm for peak detection in genomic data. Preprint
arXiv:1703.03352. Under review at {\it Annals of Applied Statistics}.

Sievert C, Cai J, VanderPlas S, Khan F, Ferris K, {\bf Hocking
  TD}. Extending ggplot2 for linked and dynamic web graphics. Under
review at {\it Journal of Computational and Graphical Statistics}.

%Venuto D,
Spanurattana S, Sugiyama M, {\bf Hocking TD}. Support vector
comparison machines. Preprint arXiv:1401.8008. 

{\bf Hocking TD}, Khare A. Learning penalty functions for changepoint
detection using elastic-net regularized accelerated failure time
models.

Narahara M, {\bf Hocking TD}, Bourque G, Yamada R, Setoh K, Matsuda F,
Lathrop M. Transcriptomic analysis of antibody responses to seasonal
influenza vaccine reveals predictive gene signatures and potential key
transcription factors.

Alirezaie N, Majewski J, {\bf Hocking TD}. A supervised machine
learning method for predicting pathogenicity of genetic variants.

\section{\sc Pre-prints}

{\bf Hocking TD}. A breakpoint detection error function for
segmentation model selection and validation. Preprint
arXiv:1509.00368.

{\bf Hocking TD}, Bourque G. PeakSegJoint: fast supervised peak
detection via joint segmentation of multiple count data
samples. Preprint arXiv:1506.01286.

{\bf Hocking TD}, Rigaill G. SegAnnot: an R package for fast
segmentation of annotated piecewise constant signals, Preprint
hal-00759129.

\section{\sc Enseignement \\aux congrès}

{\bf Hocking TD}, Killick R. Introduction to optimal changepoint detection algorithms, {\it useR} 2017.

{\bf Hocking TD}, Ekstr\o m CT. Understanding and creating interactive
graphics, {\it useR} 2016.

\section{\sc Conférences invitées}

Universit\'e Laval Centre for Big Data Research (2016), McGill
Barbados epigenomics workshop (2015), Sapporo Japan Workshop on
Machine Learning and Applications to Biology (2013), Google Research
New York (2012), Universit\'e Rennes (2012), Universit\'e Angers
(2012), INRIA Lille (2012), Institut de Biologie de Lille (2011).

\section{\sc Service Professionnel}

Président du comité d'organisation pour le congrès « R à Montreal 2018 » .

Co-administrateur et mentor pour R dans Google Summer of Code.

Comité de lecture : International Conference on Machine Learning (ICML),
Advances in Neural Infromation Processing Systems (NIPS), Journal of
Machine Learning Research (JMLR), Artificial Intelligence Review,
Journal of Computational and Graphical Statistics (JCGS), R Journal,
Bioinformatics, PLOS Computational Biology, BMC Bioinformatics, IEEE
Transactions on Pattern Analysis and Machine Intelligence.

\section{\sc Logiciels libres} 

{\bf R}: clusterpath, directlabels, animint, plotly, sublogo,
inlinedocs, quadmod, bams, neuroblastoma, breakpointError, SegAnnot,
rankSVMcompare, gganim, animint, requireGitHub, WeightedROC, revector,
PeakError, PeakSegDP, PeakSegJoint, PeakSegOptimal, memtime,
namedCapture, fpop, \mbox{penaltyLearning}, mmit, iregnet.

{\bf Python}: str.extractall in pandas, annotate\_regions, SegAnnDB.

\begin{comment}
\pagebreak

\section{\sc Références}

\noindent {\bf Francis Bach}\\
Scientific Leader,
SIERRA Project-Team\\
%Department of Computer Science\\
Institut National de Recherche en Informatique et en Automatique\\
%Phone: (604)822-9878\\
E-mail: francis.bach@ens.fr\\
Webpage: \url{http://www.di.ens.fr/~fbach}

\noindent {\bf Michael Friedlander}\\
Associate Professor,
Department of Computer Science\\
University of British Columbia\\
%Phone: (604)822-9878\\
E-mail: mpf@cs.ubc.ca\\
Webpage: http://www.cs.ubc.ca/$\sim$mpf

\noindent {\bf Kevin Murphy}\\
Research Scientist,
%Department of Computer Science, Department of Statistics\\
%University of British Columbia\\
Google\\
%Phone: (604)822-9878\\
E-mail: murphyk@gmail.com\\
Webpage: http://www.cs.ubc.ca/$\sim$murphyk


%\noindent {\bf R\'{o}mer Rosales}\\
%Research Scientist\\
%CAD and Knowledge Systems\\
%Siemens Medial Solutions USA, Inc.\\
%%Phone: (610)448 4793\\
%E-mail: romer.rosales@gmail.com\\
%Webpage: http://people.csail.mit.edu/romer/\\
%
%\noindent {\bf Glenn Fung}\\
%Research Scientist\\
%CAD and Knowledge Systems\\
%Siemens Medial Solutions USA, Inc.\\
%E-mail: glenn.fung@siemens.com\\
%Webpage: http://pages.cs.wisc.edu/$\sim$gfung/\\
%
\noindent {\bf Russell Greiner}\\
 Professor, Department of Computing Science\\
University of Alberta\\
%Phone: (604)822-9878\\
E-mail: greiner@cs.ualberta.ca\\
Webpage: http://webdocs.cs.ualberta.ca/$\sim$greiner/
\end{comment}
\end{resume}
\end{document}




